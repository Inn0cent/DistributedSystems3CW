\documentclass[11pt]{article}

\title{\vspace{-2.5cm}Computational Methodologies: Distributed Computing}
\author{bcng57}
\date{}

\usepackage{amsmath}
\usepackage{amssymb}
\usepackage[headheight=0pt, headsep=0pt, textheight = 20cm]{geometry}
\usepackage{listings}
\usepackage{xcolor}
  
\setlength{\parindent}{0pt}

\renewcommand{\thesection}{Question \arabic{section}}

\renewcommand{\thesubsection}{(\alph{subsection})}

\definecolor{codegreen}{rgb}{0,0.5,0}
\definecolor{codegray}{gray}{0.3}
\definecolor{codered}{rgb}{0.7,0,0}
\definecolor{backcolour}{rgb}{1, 1, 1}
\definecolor{test}{HTML}{ffffff}

\lstdefinestyle{distributed}{
    backgroundcolor=\color{backcolour},   
    commentstyle=\em\color{codered},
    keywordstyle=\color{codegreen},
    numberstyle=\tiny\color{codegray},
    stringstyle=\color{codepurple},
    basicstyle=\ttfamily\footnotesize,
    breakatwhitespace=false,         
    breaklines=true,                 
    captionpos=b,                    
    keepspaces=true,                 
    numbers=left,                    
    numbersep=5pt,                  
    showspaces=false,                
    showstringspaces=false,
    showtabs=false,                  
    tabsize=2,
    mathescape,
    morekeywords={upon, if, then, else, from, send, to, terminate},
    comment=[l]{//}
}
 
\lstset{style=distributed}

\begin{document}

\maketitle

\section{}

Let $G=(V_G, E_G)$ be a connected topology graph with specified root $p_r\in V_G$. Let $D=\max\{dist(p_i,p_j)\ |\ p_i,p_j\in V_G\}$, be the diameter of the graph.

\subsection{}

\textbf{Theorem}\\
The time complexity of the synchronous model of the Flooding algorithm on $G$ is $O(D)$.\\

\textbf{Proof}\\
In the worst case the distance from $p_r$ to some $p_i \in V_G$ is $D$ as this is the furthest any message can travel from $p_r$ without reaching an already visited node.
\begin{itemize}
\item In this case it takes at most $D$ rounds for $\langle M \rangle$ to reach $p_i$.
\item This is because $\langle M \rangle$ must be passed along each of the $D$ nodes on the path from $p_r$ to $p_i$ and  the message can only be transferred once per round.
\item As $p_i$ is the furthest node from $p_r$, after $D$ rounds every node will have received $\langle M \rangle$.
\end{itemize}
Once a node has received $\langle M \rangle$, each node responds in one round. Therefore, after $D+1$ rounds every node will have responded with either $\langle parent \rangle$ or $\langle already \rangle$. As all nodes have responded, each node will also have terminated.\\
Therefore, the overall time complexity is $D+1=O(D)$ %\quad $\square$

\newpage

\subsection{}

\textbf{Theorem}\\
The time complexity of the asynchronous model of the Flooding algorithm on $G$ is $O(D)$\\

\textbf{Proof}\\
Assume that the distance from $p_r$ to some $p_i \in V_G$ is $D$.\\
Let $P$ be a path of length $D$ between $p_r$ and $p_i$.\\
Suppose there is a maximum delay on every edge in $G$.\\
Then the total max time for $\langle M \rangle$ to get from $p_r$ to $p_i$ is $D$.\\
As $p_i$ is the furthest any node can be from $p_r$, every node in $V_G$ will have received $\langle M \rangle$ after $D$ maximum edge delays. \\
Once a node has received $\langle M \rangle$, it responds immediately. This response takes at most the maximum edge delay. Once every node has received a response from all its neighbours (excluding the parent), it terminates.\\
Overall, all nodes have responded and terminated in at most $D+1$ edge delays.\\
Therefore the overall time complexity is $O(D)$.


\section{}

\subsection{}

We shall prove this by contradiction.
First, assume that there is some uniform synchronous algorithm ,$A$, that correctly computes the AND of the input bits. 

Run $A$ on a ring where all input bits are 1. In any round $i$, the states of all the processors are the same. Therefore, there must be some round $r$ where all the processors terminate.

Now run $A$ on a ring with $2(r+1)$ processors, where processor $p_0$ has input bit 0 and the remaining processors have input bit 1. For $A$ to be correct some message originating from $p_0$ must reach all the nodes in the ring, however, it takes $r+1$ rounds for the message to reach processor $p_{r+1}$. As $p_{r+1}$ has initial bit 1, it will terminate in $r$ rounds with its result as 1. This is incorrect, so $A$ cannot be correct which is a contradiction. Therefore, a uniform synchronous algorithm for this problem does not exist. 


\subsection{}

Code for each node. Initially, $asleep =$ \texttt{true} for all nodes. $in$ is the binary input, $n$ is the total number of nodes

\begin{lstlisting}
upon receiving no message :
	if $asleep$ then
		$asleep$ = false
		send $\langle in, 0 \rangle$ to left

upon receiving $\langle bit, count \rangle$ from right :
	if $count$ == $n$:  // If visited nodes == total number of nodes, end
		result = $bit$
		terminate
	else:
		send $\langle bit\ \&\ in, count + 1 \rangle$ to left

\end{lstlisting}

\textbf{Plain English description}

Each processor sends a message to the left with its input bit and a counter that starts at 0.\\
Upon receiving a message of this form, a processor compares the counter to $n$.\\
\begin{itemize}
\item If the counter is equal to $n$, then its original message has been passed around the ring, so it can set the result to the received bit and terminate
\item If the counter is not equal to $n$, send a message to the left containing the received bit AND input bit, and the counter incremented by 1.
\end{itemize}
Each processor sends a message that must be forwarded to all $n$ of the processors in the ring. This means that in total $n^2$ messages are sent, so the message complexity is $O(n^2)$.\\

\textbf{Possibly more formal version}

Each processor $p_i$ sends a message $\langle b_i, c_i \rangle$ to the right with its input bit $b_i$ and a counter $c_i$ that starts at 0.\\
Upon receiving a message $\langle b_{j-1}, c_{j-1} \rangle$, a processor $p_j$ compares the $c_{j-1}$ to $n$.\\
\begin{itemize}
\item If $c_{j-1}$ is equal to $n$, then its original message has been passed around the ring, so it can set the result to $b_{j-1}$ and terminate
\item If $c_{j-1}$ is not equal to $n$, send a message $\langle b_{j-1} \text{ AND } b_j, c_{j-1} + 1 \rangle$ to the right.
\end{itemize}
Each processor sends a message that must be forwarded to all $n$ of the processors in the ring. This means that in total $n^2$ messages are sent, so the message complexity is $O(n^2)$.


\subsection{}

Code for each node. Initially, $asleep =$ \texttt{true} for all nodes. $in$ is the binary input, $n$ is the total number of nodes.\\
This algorithm has $\lfloor n/2 \rfloor$ rounds numbered $1,\dots,\lfloor n/2 \rfloor$.

\begin{lstlisting}
upon receiving no message :
	if $in$ == 0 then
		result = 0
		send $\langle terminate \rangle$ to left and right
		terminate
	else if $asleep$ then
		$asleep$ = false

upon receiving $\langle terminate \rangle$ from left (resp., right):
	result = 0
	send $\langle terminate \rangle$ to right (resp., left)
	terminate
	
if no messaged received by round $\lfloor n/2 \rfloor$:
	result = 1
	terminate

\end{lstlisting}

\textbf{Plain English description}
If the input bit of a processor is 0, then the processor sets its result to be 0, sends a $\langle terminate \rangle$ message in both directions and terminates.\\
A processor with input bit 1, waits for  $\lfloor n/2 \rfloor$ rounds. If it receives a $\langle terminate \rangle$ in this time, then it sets its result to 0, forwards the $\langle terminate \rangle$ message and terminates.\\
If no message is received within $\lfloor n/2 \rfloor$ rounds, then it sets its result to 1 and terminates.\\

In the worst case, where all the processors have 0 as an input bit, they all send 2 $\langle terminate \rangle$ messages to the left and right. This means that in total $2n$ messages are sent, so overall the message complexity is $O(n)$.
\end{document}


























